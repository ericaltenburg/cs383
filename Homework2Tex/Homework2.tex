% Homework Template for CS-383
% Eric Altenburg


\documentclass[11pt,letterpaper]{article}
\usepackage{fullpage}
\usepackage[top=2cm, bottom=4.5cm, left=2.5cm, right=2.5cm]{geometry}
\usepackage{amsmath,amsthm,amsfonts,amssymb,amscd}
\usepackage{lastpage}
\usepackage{enumerate}
\usepackage{fancyhdr}
\usepackage{mathrsfs}
\usepackage{xcolor}
\usepackage{graphicx}
\usepackage{listings}
\usepackage{hyperref}
\usepackage{lstautogobble}
\usepackage{enumitem}

\hypersetup{
  colorlinks=true,
  linkcolor=blue,
  linkbordercolor={0 0 1}
}
 
 
\renewcommand\lstlistingname{Algorithm}
\renewcommand\lstlistlistingname{Algorithms}
\def\lstlistingautorefname{Alg.}


\lstdefinestyle{Python}{
    language        = Python,
    frame           = lines, 
    basicstyle      = \footnotesize,
    keywordstyle    = \color{blue},
    stringstyle     = \color{green},
    commentstyle    = \color{red}\ttfamily
}


\setlength{\parindent}{0.0in}
\setlength{\parskip}{0.05in}


% Edit these as appropriate
\newcommand\course{CS-383}
\newcommand\hwnumber{2}                  			% <-- homework number
\newcommand\Name{Eric Altenburg}          		% <-- Name of person #1
\newcommand\ProfessorName{Prof. Peyrovian}	% <-- Professor name
\newcommand\setProbNum[1]{\setcounter{enumi}{\numexpr#1-1\relax}}

% For Code
\usepackage{listings}
\usepackage{color}


\definecolor{dkgreen}{rgb}{0,0.6,0}
\definecolor{gray}{rgb}{0.5,0.5,0.5}
\definecolor{mauve}{rgb}{0.58,0,0.82}


\lstset{frame=tb,
language=C,
aboveskip=3mm,
belowskip=3mm,
showstringspaces=false,
columns=flexible,
basicstyle={\small\ttfamily},
numbers=none,
numberstyle=\tiny\color{gray},
keywordstyle=\color{blue},
commentstyle=\color{dkgreen},
stringstyle=\color{mauve},
breaklines=true,
breakatwhitespace=true,
tabsize=3,
autogobble = true
}


\pagestyle{fancyplain}
\headheight 35pt
\lhead{\Name}
\lhead{\Name \\ \today}
\chead{\textbf{\Large Homework \hwnumber}}
\rhead{\ProfessorName \\ \course}
\lfoot{}
\cfoot{}
\rfoot{\small\thepage}
\headsep 1.5em


\begin{document}


I pledge my honor that I have abided by the Stevens Honor System. -Eric Altenburg


% This is the current chapter you are working on (#X.1)
\section*{Chapter 2}

% This will begin the actual things you are working on (#X.Y)
\begin{enumerate}
	% 2.1
	\setProbNum{1}
	\item 
	// Assign a temp register to X9\\
	SUB		X9, X2, \#5\\
	ADD		X0, X1, X9
	
	% 2.3
	\setProbNum{3}
	\item
	// Assign a temp register to X9\\
	SUB		X9, X3, X4\\
	LSL 		X9, X9, \#3\\
	ADD		X9, X9, X6\\
	LDUR	X10, $[X9, \#0$]\\
	STUR	X10, $[X7, \#64$]
	
	%2.4
	\setProbNum{4}
	\item
	\begin{lstlisting} [frame=none, language=C]
		B[g] = A[f] + A[f+1];
	\end{lstlisting}
	
	%2.5
	\setProbNum{5}
	\item
	LSL 		X9, X0, \#3\\
	ADD 	X9, X6, X9\\
	LSL		X10, X1, \#3\\
	ADD 	X10, X7, X10\\
	LDUR 	X0, $[X9, \#0$]\\\\
	
	LDUR	X9, $[X9, \#8$] // This is the combined LEGv8 code\\
	ADD 	X9, X9, X0\\
	STUR 	X9, $[X10, \#0$] 
	
	%2.9
	\setProbNum{9}
	\item
	\begin{lstlisting}[frame=none, language=C]
		f = &A[0] + &A[0];
	\end{lstlisting}
	
	%2.10
	\setProbNum{10}
	\item
	ADDI	X9, X6, \#8\\
	\begin{tabular}{| c | c | c | c |}
		\hline
		opcode & immediate & Rn & Rd\\
		\hline
		1001000100 & 000000001000 & 00110 & 01001\\
		\hline
		580 & 8 & 6 & 9\\
		\hline
	\end{tabular}	
	
	ADD		X10, X6, XZR\\
	\begin{tabular}{|c|c|c|c|c|}
		\hline
		opcode & Rm & shamt & Rn & Rd\\
		\hline
		10001011000 & 11111 & 000000 & 00110 & 01010\\
		\hline
		1112 & 31 & 0 & 6 & 10\\
		\hline
	\end{tabular}
	
	STUR	X10, $[X9, \#0$]\\
	\begin{tabular}{|c|c|c|c|c|}
		\hline
		opcode & address & op2 & Rn & Rt\\
		\hline
		11111000000 & 000000000 & 00 & 01001 & 01010\\
		\hline
		1984 & 0 & 0 & 9 & 10\\
		\hline
	\end{tabular}
	
	LDUR	X9, $[X9, \#0$]\\
	\begin{tabular}{|c|c|c|c|c|}
		\hline
		opcode & address & op2 & Rn & Rt\\
		\hline
		11111000010 & 000000000 & 00 & 01001 & 01001\\
		\hline
		1986 & 0 & 0 & 9 & 9\\
		\hline
	\end{tabular}

	ADD		X0, X9, X10\\
	\begin{tabular}{|c|c|c|c|c|}
		\hline
		opcode & Rm & shamt & Rn & Rd\\
		\hline
		10001011000 & 01010 & 000000 & 01001 & 00000\\
		\hline
		1112 & 10 & 0 & 9 & 0\\
		\hline
	\end{tabular}
	
	%2.22
	\setProbNum{22}
	\item
	$X1 = 2$
	
	%2.25
	\setProbNum{25}
	\item
		\begin{enumerate}[label=(\arabic*)]
			%1
			\setProbNum{1}
			\item 
			$X0 = 20$
		\end{enumerate}
	
	%2.41
	\setProbNum{41}
	\item
		All numbers are in terms of $10^6$, however, in the following equations, I removed it to make them more readable.
		\begin{enumerate}[label=(\arabic*)]
			%1
			\setProbNum{1}
			\item
				\begin{align*}
					CPI_{0} &= 1* \frac{500}{900} + 10 * \frac{300}{900} + 3 * \frac{100}{900}\\
					&= \frac{500}{900} + \frac{3000}{900} + \frac{300}{900}\\
					&= \frac{3800}{900}\\
					&= 4.\overline{2}\\
					Arithmetic\; Instructions &= 0.75 * 500\\
					&= 375\\
					CPU_{0} &= CPI_{0} * f * 900\\
					&= 3800f\\
					CPI_{new} &= 1* \frac{375}{775} + 10 * \frac{300}{775} + 3 * \frac{100}{775}\\
					&= \frac{375}{775} + \frac{3000}{775} + \frac{300}{775}\\
					&= \frac{3675}{775}\\
					&= 4.74\\
					CPU_{new} &= CPI_{new} * f * 775 * 1.10\\
					&= 4042.5f
				\end{align*}
				This would not be a good idea as the total execution time has increased from what it used to be.
			
			%2
			\setProbNum{2}
			\item
				\begin{align*}
					Arithmetic\; CPI_{new} &= 1*\frac{1}{2} = \frac{1}{2}\\
					CPI_{new} &= \frac{1}{2} * \frac{500}{900} + 10 * \frac{300}{900} + 3 * \frac{100}{900}\\
					&= 3.9\overline{4}\\
					Speedup_{formula} &= \frac{high}{low}\\
					Speedup_{arithmetic} &= \frac{4.\overline{2}}{3.9\overline{4}} = 1.07\\
					7\% \;increase\\\\
					Arithmetic\; CPI_{10\; times} &= 1* \frac{1}{10} = \frac{1}{10}\\
					CPI_{10\; times} &= \frac{1}{10} * \frac{500}{900} + 10 * \frac{300}{900} + 3 * \frac{100}{900}\\
					&= 3.7\overline{2}\\
					Speedup_{10\; times} &= \frac{4.\overline{2}}{3.7\overline{2}} = 1.13\\
					13\% \;increase
				\end{align*}
		\end{enumerate}
	
	%2.42
	\setProbNum{42}
	\item
		\begin{enumerate}[label=(\arabic*)]
			%1
			\setProbNum{1}
			\item
				\begin{align*}
					CPI_{avg} &= 0.7 * 2 + 0.1 * 6 + 0.2 * 3\\
					&= 2.6
				\end{align*}
				
			%2
			\setProbNum{2}
			\item
				\begin{align*}
					IPC_{0} &= 1/2.6 = 0.3846\\
					IPC_{new} &= IPC_{0} * 1.25 = .4808\\
					CPI_{new} &= \frac{1}{IPC_{new}} = 2.08\\\\
					CPI_{new} &= 0.7 * Arithmetic\; Cycles + 0.1 * 6 + 0.2 * 3\\
					2.08 &= 0.7*Arithmetic\; Cycles + 1.2\\
					0.7 * Arithmetic\; Cycles &= 0.88\\
					Arithmetic\; Cycles &= 1.2571 \approx 1
				\end{align*}
				It doesn't make sense to have a float for the number of cycles, so it was approximated to 1.
		\end{enumerate}


\end{enumerate}


\end{document}